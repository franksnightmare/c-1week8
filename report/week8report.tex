\documentclass[11pt]{article}

\usepackage{times}
\usepackage[english]{babel}

% -----------------------------------------------
% especially use this for you code
% -----------------------------------------------

\usepackage{courier}
\usepackage{listings}
\usepackage{color}
\usepackage{tabularx}
\usepackage{graphicx}

\definecolor{Gray}{gray}{0.95}

\definecolor{mygreen}{rgb}{0,0.6,0}
\definecolor{mygray}{rgb}{0.5,0.5,0.5}
\definecolor{mymauve}{rgb}{0.58,0,0.82}

\lstset{language=C++,
	basicstyle = \normalsize\ttfamily,   % the size and fonts that are used
	tabsize = 2,                    % sets default tabsize
	breaklines = true,              % sets automatic line breaking
	keywordstyle=\color{blue}\ttfamily,
	stringstyle=\color{red}\ttfamily,
	commentstyle=\color{mygreen}\ttfamily,
	numbers=left,
	keepspaces=true,
	showspaces=false,
	showstringspaces=false,
}

\begin{document}

\title{Programming in C/C++ \\
       Exercises set eight: Overloading
}
\date{\today}
\author{Christiaan Steenkist \\
Diego Ribas Gomes \\
Jaime Betancor Valado \\
Remco Bos \\
}

\maketitle

\section*{The \texttt{Matrix} class}
We have merged all the different matrices together to form the one true \texttt{Matrix}.
Here is the header file for that class so we do not have to print it time and time again.
\lstinputlisting[caption = matrix.ih]{src/matrix/matrix.ih}
\lstinputlisting[caption = matrix.h]{src/matrix/matrix.h}

\section*{Exercise 67, the index operator}
We implemented the index operator for the \texttt{Matrix} class.
It returns a pointer to the first double of a row in the matrix.
We also made a const variant of this function.

\subsection*{Code listings}
%\lstinputlisting[caption = operator\verbatim{[]}.cc]{src/matrix/operator[].cc}
%\lstinputlisting[caption = operator\verbatim{[]}const.cc]{src/matrix/operator[]const.cc}

\section*{Exercise 68, addition operators}
We implemented the "\texttt{+}" and "\texttt{+=}" operators for our matrix class.
Matrices can only be added if the dimensions are equal.

\subsection*{Code listings}
\lstinputlisting[caption = operator+.cc]{src/matrix/operator+.cc}
\lstinputlisting[caption = operator\texttt{+=}.cc]{src/matrix/operator+=.cc}

\section*{Exercise 69, insertion and extraction}
We implemented the insertion and extraction for the \texttt{Matrix} class.
For insertion we used a single function that is used on matrices.
For extraction we made a matrix function and a proxy function.
The proxies are made using the overloaded parenthesis operator and the different parenthesis operators give the proxy different settings.
The main function shows the slightly changed syntax compared to the code in the assignment.

\subsection*{Code listings}
\lstinputlisting[caption = main.cc]{src/a69/main.cc}

\subsubsection*{Parenthesis operators and Proxy}
\lstinputlisting[caption = proxy.cc]{src/matrix/proxy.cc}
\lstinputlisting[caption = poperator1.cc]{src/matrix/poperator1.cc}
\lstinputlisting[caption = poperator2.cc]{src/matrix/poperator2.cc}
\lstinputlisting[caption = poperator3.cc]{src/matrix/poperator3.cc}
\lstinputlisting[caption = poperator3b.cc]{src/matrix/poperator3b.cc}

\subsubsection*{Insertion and extraction}
\lstinputlisting[caption = inserter.cc]{src/matrix/inserter.cc}
\lstinputlisting[caption = extractor.cc]{src/matrix/extractor.cc}
\lstinputlisting[caption = proxyextractor.cc]{src/matrix/proxyextractor.cc}

\section*{Exercise 70, equality and inequality operators}
We overloaded the "\texttt{==}" and "\texttt{!=}" operators for both the \texttt{Strings} and the \texttt{Matrix} classes.

\subsection*{Code listings}
\subsubsection*{\texttt{Strings}}
\lstinputlisting[caption = strings.h]{src/a70/strings/strings.h}
\lstinputlisting[caption = operator\texttt{==}.cc]{src/a70/strings/operator==.cc}
\lstinputlisting[caption = operator\texttt{!=}.cc]{src/a70/strings/operator!=.cc}

\subsubsection*{\texttt{Matrix}}
\lstinputlisting[caption = operator\texttt{==}.cc]{src/matrix/operator==.cc}
\lstinputlisting[caption = operator\texttt{!=}.cc]{src/matrix/operator!=.cc}

\section*{Exercise 71, explicit constructors}
What constructors are potential candidates for 'explicit'?
The answer is constructors with only one parameter.
They can act like implicit converters, so to avoid this we should use the keyword "explicit" to allow only explicit conversions.

\subsection*{Code listing}
\lstinputlisting[caption = strings.h]{src/a71/strings.h}

\end{document}
